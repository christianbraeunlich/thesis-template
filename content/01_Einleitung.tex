\chapterimage{media/headerLogo.png}
\chapter{Einführung}
\label{cha:einfuehrung}

\epigraph{Many of life's failures are people who did not realize\\how close they were to success when they gave up.}{Thomas A. Edison}

\section{Einleitung}
\label{sec:einleitung}

How to cite \cite{yt:online}.

\lipsum[9]

\section{Motivation}
\label{sec:motivation}

\begin{figure}[!ht]
    \centering
    \includegraphics[width=0.9\linewidth]{media/placeholder.png}
    \caption[Platzhalter]{Platzhalter}
    \label{fig:srs_process}
\end{figure}

\lipsum[9]

\section{Ziel der Arbeit}
\label{sec:zielderarbeit}

\lipsum[1]

%\begin{enumerate}[label=\textbf{S.\arabic*},ref=S.\arabic*]
\begin{enumerate}[label=\textit{\textbf{Forschungsfrage \#\arabic*}}, leftmargin=4.25cm, resume]
    \item \label{FF1} \glqq ?\grqq{}
    \item \label{FF2} \glqq ?\grqq{}
    \item \label{FF3} \glqq ?\grqq{}
\end{enumerate}

\lipsum[1]

\section{Abgrenzungen}
\label{sec:abgrenzungen}

\lipsum[2]

\section{Vergleichende Arbeiten}
\label{cha:relatedwork}

\lipsum[2]

\clearpage

\textbf{\autoref{cha:grundlagennlp}: \nameref{cha:grundlagennlp}}\\
\lipsum[2]

\textbf{\autoref{cha:durchfuehrungfeature}: \nameref{cha:durchfuehrungfeature}}\\
\lipsum[2]

\textbf{\autoref{cha:implementierung}: \nameref{cha:implementierung}}\\
\lipsum[2]

\textbf{\autoref{cha:ergebnisse}: \nameref{cha:ergebnisse}}\\
\lipsum[2]

\textbf{\autoref{cha:fazit}: \nameref{cha:fazit}}\\
\lipsum[2]
