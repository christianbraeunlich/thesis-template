%\pdfcompresslevel=9

%% PDF-Optionen -------------------------------------------------------------
\usepackage[
extension=pdf,
pdflang={de},
pdftitle={\titlename},
pdfauthor={\authorname},
pdfcreator={\authorname},
pdfsubject={\titlename},
pdfkeywords={\keywords},
linktoc=all,
bookmarks,                      % Lesezeichen beim Betrachten anzeigen
bookmarksopen=true,             % alle Lesezeichen ausklappen
bookmarksopenlevel=1,           % Festlegen der Tiefe der Bookmarks
%bookmarksnumbered,              % Abschnittsnummer anzeigen
%pdfpagelabels,                  % zur korrekten Erstellung der Bookmarks
%plainpages,                     % zur korrekten Erstellung der Bookmarks
%
%backref,                        % Fügt im Literaturverzeichnis einen Rücksprungspunkt ein
%backref=section,                % ???
%backref=slide,                  % ??? 
%
%pdfmenubar,                     % Acrobat’s Menüleiste
%pdftoolbar,                     % Anzeigen der Acrobat’s Werkzeugleiste
%pdffitwindow,                   % Resize document to fit document size
%pdfnewwindow=true,              % links in new PDF window
%pdfpagelabels,                  % ???
%pdfpagelayout=TwoPageRight,
%pdfpagemode=UseThumbs,
%pdfview=Fit,
pdfview={XYZ null null 1},
%pdfstartview=Fit,
%pagebackref,                    % ???
%
%colorlinks,                     % false: boxed links; true: colored links
%linktocpage,                     % Seitenzahlen anstatt Text im Inhaltsverzeichnis verlinken
%linkcolor=red,                  % color of internal links (change box color with linkbordercolor)
%linkbordercolor={1 0 0},        % color of frame around internal links (if colorlinks=false)
%citecolor=green,                % color of links to bibliography
%citebordercolor={1 0 0},        % color of frame around citations
%urlcolor=cyan,                  % color of external links
%urlbordercolor={1 0 0}          % color of frame around URL links
%
%filecolor=magenta,              % color of file links
%menucolor=red,                  % Farbe der Acrobat-Menüpunkte
%
%pdfborderstyle={/S/U/W 1},
%pdfborder={6 6 6},
%anchorcolor=black,% Ankertext
%hypertexnames=false            % zur korrekten Erstellung der Bookmarks
]{hyperref}

\usepackage{etoolbox}

\makeatletter
\pretocmd{\contentsline}
  {\patchcmd{\cftdotfill}
     {\leaders}
     {\hyper@linkstart{link}{#4}\leaders}
     {}
     {}%
   \patchcmd{\cftdotfill}
     {\hfill}
     {\hfill\hyper@linkend}
     {}
     {}}
  {}
  {}
\makeatother