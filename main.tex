%%%%%%%%%%%%%%%%%%%%%%%%%%%%%%%%%%%%%%%%%%%%%%%%%%%%%%%%%%%%%%%%%%%%%%%%%%%%
%               _   _            
%              | | | |           
%          ___ | |_| |_ ___ _ __ 
%         / _ \| __| __/ _ \ '__|
%        | (_) | |_| ||  __/ |   
%         \___/ \__|\__\___|_|   
%
%         .-"""-.
%        /      o\
%       |    o   0).-.
%       |       .-;(_/     .-.
%        \     /  /)).---._|  `\   ,
%         '.  '  /((       `'-./ _/|
%           \  .'  )        .-.;`  /
%            '.             |  `\-'
%              '._        -'    /
%                 ``""--`------`
%
%%%%%%%%%%%%%%%%%%%%%%%%%%%%%%%%%%%%%%%%%%%%%%%%%%%%%%%%%%%%%%%%%%%%%%%%%%%%
%%%%%%%%%%%              Thesis-Dokumenten-Vorlage               %%%%%%%%%%%
%%%%%%%%%%%%%%%%%%%%%%%%%%%%%%%%%%%%%%%%%%%%%%%%%%%%%%%%%%%%%%%%%%%%%%%%%%%%
%
% written in LaTeX with Overleaf
% 
% Author: Christian Bräunlich
% Date: 01.06.2020
% Organization: Hochschule Flensburg University of Applied Sciences 
% 
% -- MIT License
%
% Copyright (c) [2020] [Christian Bräunlich]
% 
% more Infos available in: LICENSE.txt
%%%%%%%%%%%%%%%%%%%%%%%%%%%%%%%%%%%%%%%%%%%%%%%%%%%%%%%%%%%%%%%%%%%%%%%%%%%%

% ***********************
% Document Configuration
% ***********************
% Benötigte Packages -------------------------------------------------------
%		Weitere Packages, die benötigt werden, sind in die Datei Packages.tex
%		"ausgelagert", um die Vorlage möglichst übersichtlich zu halten.
%
% --------------------------------------------------------------------------
%%%
%%% Benutzer Einstellungen
% **** User-Informations **************************************
% 		- Definition of global Parameters.
%
% *************************************************************
\newcommand{\art}{Bachelor-Thesis}
\newcommand{\titlename}{<insert the title for this document here>}
\newcommand{\subtitlename}{<insert subtitle>}
\newcommand{\authorname}{<Author Name>}
\newcommand{\authoraddress}{<insert address>}
\newcommand{\authorlocation}{<insert location>}
\newcommand{\studiengang}{<study course>}
\newcommand{\matrikelnr}{123456}
\newcommand{\erstgutachter}{<insert reviewer No.1>}
\newcommand{\zweitgutachter}{<insert reviewer No.2>}
\newcommand{\institutname}{<insert institution name>}
\newcommand{\institutaddress}{<insert institution address>}
\newcommand{\institutlocation}{<insert instituion location>}
\newcommand{\akademischergrad}{Bachelor of Science (B.Sc.)}
\newcommand{\ausgabedatum}{<date of start>}
\newcommand{\abgabedatum}{<date of finish>}

\newcommand{\keywords}{insert, keywords, seperated, by, comma}

% Eigene Befehle und typographische Auszeichnungen für diese
\newcommand{\todo}[1]{\textbf{\textsc{\textcolor{red}{(TODO: #1)}}}}

\newcommand{\AutorZ}[1]{\textsc{#1}}
\newcommand{\Autor}[1]{\AutorZ{\citeauthor{#1}}}

\newcommand{\NeuerBegriff}[1]{\textbf{#1}}

\newcommand{\Fachbegriff}[1]{\textit{#1}}
\newcommand{\Prozess}[1]{\textit{#1}}
\newcommand{\Webservice}[1]{\textit{#1}}

\newcommand{\Eingabe}[1]{\texttt{#1}}
\newcommand{\Code}[1]{\texttt{#1}}
\newcommand{\Datei}[1]{\texttt{#1}}

\newcommand{\Datentyp}[1]{\textsf{#1}}
\newcommand{\XMLElement}[1]{\textsf{#1}}

% Abkürzungen mit korrektem Leerraum
\newcommand{\vgl}{Vgl.\ }
\newcommand{\ua}{\mbox{u.\,a.\ }}
\newcommand{\zB}{\mbox{z.\,B.\ }}
\newcommand{\bs}{$\backslash$}
%%%
%%% Dokumenten Klasse
%%%%%%%%%%%%%%%%%%%%%%%%%%%%%%%%%%%%%%%%%%%%%%%%%%%%%%%%%%%%
%%%%%%%%%%%           Document-Class             %%%%%%%%%%%
%%%%%%%%%%%%%%%%%%%%%%%%%%%%%%%%%%%%%%%%%%%%%%%%%%%%%%%%%%%%
%
% ____ Scrreprt-Package ____________________________________
%
% https://...
%
\documentclass[
	11pt,				% Schriftgröße
	german,				% für Umlaute, Silbentrennung etc.
	a4paper,         	% Papierformat
	%oneside,			% einseitiges Dokument
	twoside,            % zweiseitiges Dokument
	openright,          % starte Kapitel auf der rechten Seite
	pointlessnumbers,   % kein Punkt hinter der Nummer
	titlepage,			% Titelseite
	parskip=half,		% Abstand zwischen Absätzen (halbe Zeile)
	headings=normal,	% Größe der Überschriften verkleinern
	liststotoc,			% Verzeichnisse im Inhaltsverzeichnis aufführen
	bibtotoc,			% Literaturverzeichnis im Inhaltsverzeichnis aufführen
	index=totoc,		% Index im Inhaltsverzeichnis aufführen
	tablecaptionabove,	% Beschriftung von Tabellen unterhalb ausgeben
	final				% Status des Dokuments (final/draft)
]{scrreprt}
%%%
%%% Paket-Import
% **** Testing Packages **************************************
% 		- Definition of global Parameters.
%
% *************************************************************

\usepackage{lipsum}
% \lipsum
% \lipsum[3-56]


% **** User-Informations **************************************
% 		- Definition of global Parameters.
%
% *************************************************************

% Umlaute ------------------------------------------------------------------
% 	Umlaute/Sonderzeichen wie äüöß direkt im Quelltext verwenden (CodePage).
%		Erlaubt automatische Trennung von Worten mit Umlauten.
% --------------------------------------------------------------------------
\usepackage{microtype} % Slightly tweak font spacing for aesthetics
\usepackage[utf8]{inputenc}
\usepackage[T1]{fontenc}
\usepackage{lmodern}
%\usepackage{ae} % "schöneres" ß
\usepackage{textcomp} % Euro-Zeichen etc.

% Anpassung an Landessprache -----------------------------------------------
% 	Verwendet globale Option german siehe \documentclass
% --------------------------------------------------------------------------
\usepackage{babel}


\usepackage{emptypage}


\usepackage{tocbasic}

% Quotation Package
\usepackage{csquotes}

% Bessere Unterstreichungen ---------------------------------------------
\usepackage[normalem]{ulem}


% Abkürzungsverzeichnis
\usepackage[acronym, toc]{glossaries}
\makeglossaries

% Grafiken -----------------------------------------------------------------
% 	Einbinden von EPS-Grafiken [draft oder final]
% 	Option [draft] bindet Bilder nicht ein - auch globale Option
% --------------------------------------------------------------------------
\usepackage[dvips,final]{graphicx}
\graphicspath{{media/}} % Dort liegen die Bilder des Dokuments

% Befehle aus AMSTeX für mathematische Symbole z.B. \boldsymbol \mathbb ----
\usepackage{amsmath,amsfonts}

%
% Zeilenumbruch bei Bildbeschreibungen
%
\setcapindent{1em}

% Für Index-Ausgabe; \printindex -------------------------------------------
\usepackage{makeidx}

% Symbolverzeichnis --------------------------------------------------------
% 	Symbolverzeichnisse bequem erstellen, beruht auf MakeIndex.
% 		makeindex.exe %Name%.nlo -s nomencl.ist -o %Name%.nls
% 	erzeugt dann das Verzeichnis. Dieser Befehl kann z.B. im TeXnicCenter
%		als Postprozessor eingetragen werden, damit er nicht ständig manuell
%		ausgeführt werden muss.
%		Die Definitionen sind ausgegliedert in die Datei Abkuerzungen.tex.
% --------------------------------------------------------------------------
\usepackage[intoc]{nomencl}
  \let\abbrev\nomenclature
  \renewcommand{\nomname}{Abkürzungsverzeichnis}
  \setlength{\nomlabelwidth}{.25\hsize}
  \renewcommand{\nomlabel}[1]{#1 \dotfill}
  \setlength{\nomitemsep}{-\parsep}

% zum Umfließen von Bildern ---------------------------------------------------------
\usepackage[vflt]{floatflt}
\usepackage{subfigure}

% Zum Einbinden von Programmcode --------------------------------------------
\usepackage{listings}
\usepackage{xcolor} 
\definecolor{hellgelb}{rgb}{1,1,0.9}
\definecolor{colKeys}{rgb}{0,0,1}
\definecolor{colIdentifier}{rgb}{0,0,0}
\definecolor{colComments}{rgb}{1,0,0}
\definecolor{colString}{rgb}{0,0.5,0}
\lstset{%
    float=hbp,%
    basicstyle=\texttt\small, %
    identifierstyle=\color{colIdentifier}, %
    keywordstyle=\color{colKeys}, %
    stringstyle=\color{colString}, %
    commentstyle=\color{colComments}, %
    columns=flexible, %
    tabsize=2, %
    frame=single, %
    extendedchars=true, %
    showspaces=false, %
    showstringspaces=false, %
    numbers=left, %
    numberstyle=\tiny, %
    breaklines=true, %
    backgroundcolor=\color{hellgelb}, %
    breakautoindent=true, %
%    captionpos=b%
}

% Lange URLs umbrechen etc. -------------------------------------------------
\usepackage{url}

%\usepackage{jurabib}
%\jurabibsetup{authorformat=smallcaps,% Autor in Kapitälchen              
%              %authorformat=year,
%              authorformat=citationreversed,% Im Zitat Vorname vorne
%              authorformat=indexed,% Autor in Index
%              authorformat=and,% Autoren mit "," und "und" abgetrennt
%              authorformat=firstnotreversed,%
%              authorformat=reducedifibidem,% Bei Verweis nur Nachname. 
%              %superscriptedition=all,% Auflage hochgestellt
%              %citefull=first,% Erstzitat voll
%              titleformat=italic,              
%              titleformat=all,
%              titleformat=colonsep,% Doppelpunkt zwischen Aut. u. Titel
%              ibidem=strict,% Ebenda pro Doppelseite
%              see,% Das zweite Argument ist optional für "Vgl." etc.
%              commabeforerest,% Komma vor Seitenzahl
%              %howcited=compare,%
%              %bibformat=ibidem,% Strich bei widerholtem Autor in BIB.
%              commabeforerest,
%              bibformat=compress,
%              pages=always,
%              %pages=format,% S. wird vorweggesetzt
%              crossref=long,% Querverweise in voller Länge
%              square,% eckige Klammern bei Zitaten
%              %oxford,
%              %chicago,
%}
%
%\AddTo\bibsgerman{% 
%\jblookforgender%
%\renewcommand*{\ibidemname}{Ebenda}%
%\renewcommand*{\ibidemmidname}{ebenda}% 
%\renewcommand*{\bibjtsep}{In: }% Vor Zeitschriften 
%\renewcommand*{\bibbtsep}{In: }% Vor Buchtitel
%\renewcommand*{\incollinname}{In: }%Nicht so ganz sauber. 
%\renewcommand*{\bibatsep}{.}% Nach Titel
%\renewcommand*{\bibbdsep}{}%Vor Datum 
%\renewcommand*{\jbaensep}{.}%
%\renewcommand*{\bibprdelim}{)}% Klammer bei Zeitschriftjahr rechts
%\renewcommand*{\bibpldelim}{(}% Klammer bei Zeitschriftjahr links
%\renewcommand*{\biblnfont}{\textsc}% Nachamen Autor im BIB
%\renewcommand*{\bibelnfont}{\textsc}% Nachamen Hg. im BIB
%\renewcommand*{\bibfnfont}{\textsc}% Vorn. Autor im BIB
%\renewcommand*{\bibefnfont}{\textsc}% Vorn. Hg. im BIB
%\renewcommand*{\jbcitationyearformat}[1]{#1}% Komma zwischen Autor und Jahr entfernen
%\def\herename{hier: }%
%\jbfirstcitepageranges% Format: S. x--z, hier y.  
%\renewcommand\bibidemSfname{\raisebox{.2em}{\rule{2.em}{.2pt}}~}%
%\renewcommand\bibidemsfname{\raisebox{.2em}{\rule{2.em}{.2pt}}~}%
%\renewcommand\bibidemPfname{\raisebox{.2em}{\rule{2.em}{.2pt}}~}%
%\renewcommand\bibidempfname{\raisebox{.2em}{\rule{2.em}{.2pt}}~}%
%\renewcommand\bibidemSmname{\raisebox{.2em}{\rule{2.em}{.2pt}}~}%
%\renewcommand\bibidemsmname{\raisebox{.2em}{\rule{2.em}{.2pt}}~}%
%\renewcommand\bibidemPmname{\raisebox{.2em}{\rule{2.em}{.2pt}}~}%
%\renewcommand\bibidempmname{\raisebox{.2em}{\rule{2.em}{.2pt}}~}%
%\renewcommand\idemSfname{Dies.}%
%\renewcommand\idemsfname{dies.}%
%\renewcommand\idemPfname{Dies.}%
%\renewcommand\idempfname{dies.}%
%\renewcommand\idemSmname{Ders.}%
%\renewcommand\idemsmname{ders.}%
%\renewcommand\idemPmname{Dies.}%
%\renewcommand\idempmname{dies.}%
%\renewcommand{\jbannoteformat}[1]{{\footnotesize\begin{quote}#1\end{quote}}}
%}%

% Paket zum sauberen Einbauen von externen PDF-Dateien -----------------
\usepackage[final]{pdfpages}

% Zum fortlaufenden Durchnummerieren der Fußnoten ---------------------------
\usepackage{chngcntr}

% Aliase für Zitate
% \defcitealias{WPProzess}{Wikipedia:~Prozess}

%\usepackage{minitoc}

% Beschriftung von Tabellen und Bildern ändern ----------------------------------------------------------
\addto\captionsngerman{
	\renewcommand{\figurename}{Abb.}
	\renewcommand{\tablename}{Tab.}
}

% Rotation von Elementen -------------------------------------------------------
\usepackage{rotating}

% für lange Tabellen
\usepackage{longtable}
\usepackage{array}
\usepackage{ragged2e}
\usepackage{lscape}

%Spaltendefinition rechtsbündig mit definierter Breite
\newcolumntype{z}[1]{>{\raggedleft\hspace{0pt}}p{#1}}

% Formatierung von Listen ändern
\usepackage{paralist}
% \setdefaultleftmargin{2.5em}{2.2em}{1.87em}{1.7em}{1em}{1em}

% Anhangsverzeichnis
%\makeatletter% --> De-TeX-FAQ
%\newcommand*{\maintoc}{% Hauptinhaltsverzeichnis
%\begingroup
%\@fileswfalse% kein neues Verzeichnis öffnen
%\renewcommand*{\appendixattoc}{% Trennanweisung im Inhaltsverzeichnis
%\value{tocdepth}=-10000 % lokal tocdepth auf sehr kleinen Wert setzen
%}%
%\tableofcontents% Verzeichnis ausgeben
%\endgroup
%}
%\newcommand*{\appendixtoc}{% Anhangsinhaltsverzeichnis
%\begingroup
%\edef\@alltocdepth{\the\value{tocdepth}}% tocdepth merken
%\setcounter{tocdepth}{-10000}% Keine Verzeichniseinträge
%\renewcommand*{\contentsname}{% Verzeichnisname ändern
%Verzeichnis der Anh\"ange}%
%\renewcommand*{\appendixattoc}{% Trennanweisung im Inhaltsverzeichnis
%\setcounter{tocdepth}{\@alltocdepth}% tocdepth wiederherstellen
%}%
%\tableofcontents% Verzeichnis ausgeben
%\setcounter{tocdepth}{\@alltocdepth}% tocdepth wiederherstellen
%\endgroup
%}
%\newcommand*{\appendixattoc}
%\g@addto@macro\appendix{% \appendix erweitern
%\if@openright\cleardoublepage\else\clearpage\fi% Neue Seite
%\addcontentsline{toc}{chapter}{\appendixname}% Eintrag ins Hauptverzeichnis
%\addtocontents{toc}{\protect\appendixattoc}% Trennanweisung in die toc-Datei
%}
%\makeatother
%%%
%%% Seiten Einstellungen
%%%%%%%%%%%%%%%%%%%%%%%%%%%%%%%%%%%%%%%%%%%%%%%%%%%%%%%%%%%%
%%%%%%%%%%%             Page-Layout              %%%%%%%%%%%
%%%%%%%%%%%%%%%%%%%%%%%%%%%%%%%%%%%%%%%%%%%%%%%%%%%%%%%%%%%%
%
% https://en.wikibooks.org/wiki/LaTeX/Page_Layout
%
% ____ Geometry-Package ____________________________________
%
% https://...
%

\makeatletter
\if@twoside%
   %%%%%%%%%%%%%%% TWO-SIDE

   \usepackage[
	paper=a4paper,          % Change to letterpaper for US letter
	inner=2.0cm,            % Inner margin
	outer=2.0cm,            % Outer margin
	bindingoffset=1.0cm,    % Binding offset
	top=1.5cm,              % Top margin
	bottom=1.5cm,           % Bottom margin
	includeheadfoot,        % Include header and footer
	footnotesep=1.25cm,     % Footnotes Spacing
    %headsep=1.25cm,        % Header Spacing
	%showframe,             % Uncomment to show how the type block is set on the page
	]{geometry}
	
	%\setlength{\marginparsep}{1.25cm}
    \setlength{\marginparwidth}{1.25cm}
    
    %\setlength{\topmargin}{0.00cm}
    %\setlength{\headheight}{0.25cm}
    %\setlength{\headsep}{0.25cm} %% Setting for gap in header and main body%%
    \setlength{\textheight}{24.0cm}
    %\setlength{\textwidth}{14.65cm}
    \setlength{\footskip}{1.25cm}
    
    %%%%%%%%%%%%%%%%%%%%%%%
\else%
   %%%%%%%%%%%%%%% ONE-SIDE

   \usepackage[
	paper=a4paper,           % Change to letterpaper for US letter
	%inner=2.0cm,            % Inner margin
	%outer=2.0cm,            % Outer margin
	%bindingoffset=1.0cm,    % Binding offset
	%top=1.5cm,              % Top margin
	%bottom=1.5cm,           % Bottom margin
	includeheadfoot,         % Include header and footer
	%footnotesep=2cm,        % Footnotes Spacing
    %headsep=2.0cm,          % Header Spacing
	%showframe,              % Uncomment to show how the type block is set on the page
	]{geometry}
\fi%  
\makeatother

%\usepackage{showframe}      				% to display the page layout

% A4 -----------------------------------------------------------------------
%		DIN-A4 paper size settings
%		
% --------------------------------------------------------------------------
\setlength{\paperheight}{29.7cm}
\setlength{\paperwidth}{21.0cm}
\setlength{\hoffset}{0.0cm}
\setlength{\voffset}{0.0cm}
%\setlength{\evensidemargin}{0.0cm}
%\setlength{\oddsidemargin}{1.27cm}

%\setlength{\marginparsep}{1.25cm}
% \setlength{\marginparwidth}{1.25cm}

%\setlength{\topmargin}{0.00cm}
%\setlength{\headheight}{0.25cm}
%\setlength{\headsep}{0.25cm} 				% Setting for gap in header and main body%%
% \setlength{\textheight}{24.0cm}
%\setlength{\textwidth}{14.65cm}
% \setlength{\footskip}{1.5cm}

\frenchspacing								% creates a little more space behind a point

%----------------------------------------------------------------------------------------
%	PENALTIES
%----------------------------------------------------------------------------------------
\doublehyphendemerits=10000     % No consecutive line hyphens
\brokenpenalty=10000            % No broken words across columns/pages
\widowpenalty=9999              % Almost no widows at bottom of page
\clubpenalty=9999               % Almost no orphans at top of page
\interfootnotelinepenalty=9999  % Almost never break footnotes

% Format source code output --------------------------------------------
\lstset{numbers=left, numberstyle=\tiny, numbersep=5pt, breaklines=true}
\lstset{emph={square}, emphstyle=\color{red}, emph={[2]root,base}, emphstyle={[2]\color{blue}}}

% Number footnotes consecutively ------------------------------------
\counterwithout{footnote}{chapter}

% Insert landscape format ------------------------------------------------------
%		Changes the orientation of the document in a specified area.
%		
%\usepackage{pdflscape}
%\begin{landscape}
%\begin{table}
%\centering
%\begin{tabular}{....}
% ...
%\end{tabular}
%\end{table}
%\end{landscape}
% --------------------------------------------------------------------------
%%%
%%% Chapter Style Einstellungen
%%%%%%%%%%%%%%%%%%%%%%%%%%%%%%%%%%%%%%%%%%%%%%%%%%%%%%%%%%%%
%%%%%%%%%%%              Bibliography            %%%%%%%%%%%
%%%%%%%%%%%%%%%%%%%%%%%%%%%%%%%%%%%%%%%%%%%%%%%%%%%%%%%%%%%%
%
% ____ Adding lines above the chapter head _________________
%

%\newcommand*{\ORIGchapterheadstartvskip}{}%
%\let\ORIGchapterheadstartvskip=\chapterheadstartvskip
%\renewcommand*{\chapterheadstartvskip}{%
%  \ORIGchapterheadstartvskip
%  {%
%    \setlength{\parskip}{0pt}%
%    \noindent\rule[.3\baselineskip]{\linewidth}{1pt}\par
%  }%
%}
 
% ____ Adding lines below the chapter head _________________
\newcommand*{\ORIGchapterheadendvskip}{}%
\let\ORIGchapterheadendvskip=\chapterheadendvskip
\renewcommand*{\chapterheadendvskip}{%
  {%
    \setlength{\parskip}{0pt}%
    \noindent\rule[.3\baselineskip]{\linewidth}{1pt}\par
  }%
  \ORIGchapterheadendvskip
}
%%%
%%% PDF Einstellungen
%%%%%%%%%%%%%%%%%%%%%%%%%%%%%%%%%%%%%%%%%%%%%%%%%%%%%%%%%%%%
%%%%%%%%%%%          PDF-Configuration           %%%%%%%%%%%
%%%%%%%%%%%%%%%%%%%%%%%%%%%%%%%%%%%%%%%%%%%%%%%%%%%%%%%%%%%%
%
% ____ Hyperref-Package Options ____________________________
%
% https://...
%
% https://www.sciencetronics.com/greenphotons/wp-content/uploads/2016/10/xcolor_names.pdf

\usepackage[
extension=pdf,
pdflang={de},
%pdfmenubar,                        % Acrobat's menu bar
%pdftoolbar,                        % Acrobat's toolbar
pdftitle={\titlename},
pdfsubject={\titlename},
pdfauthor={\authorname},
pdfkeywords={\keywords},
pdfcreator={\authorname},
pdfproducer={LaTeX with hyperref},  % LaTeX with hyperref
linktoc=all,
bookmarks,                          % Show bookmarks when viewing
bookmarksopen=true,                 % expand all bookmarks
bookmarksopenlevel=0,               % Set the depth of the bookmarks
%bookmarksnumbered,                 % Show section number
%pdfpagelabels,                     % for the correct creation of the bookmarks
%plainpages,                        % for the correct creation of the bookmarks
%
pdffitwindow,                       % Resize document to fit document size
%pdfnewwindow=true,                 % links in new PDF window
%pdfpagelabels,                     % ???
pdfpagelayout=TwoPageRight,
pdfpagemode=UseThumbs,
%pdfview=Fit,
pdfview={XYZ null null 1},
%pdfstartview=Fit,
%
%colorlinks,                        % false: boxed links; true: colored links
%linktocpage,                       % Link page numbers instead of text in the table of contents
linkcolor=SpringGreen4,             % color of internal links (change box color with linkbordercolor)
%linkbordercolor={1 0 0},           % color of frame around internal links (if colorlinks=false)
citecolor=DodgerBlue4,              % color of links to bibliography
%citebordercolor={1 0 0},           % color of frame around citations
urlcolor=RoyalBlue3,                % color of external links
%urlbordercolor={1 0 0}             % color of frame around URL links
%
%filecolor=magenta,                 % color of file links
%menucolor=red,                     % Acrobat menu item color
%
%pdfborderstyle={/S/U/W 1},
%pdfborder={6 6 6},
%anchorcolor=black,% Ankertext
%hypertexnames=false                % for the correct creation of the bookmarks
]{hyperref}


% ______PDF-Table of Contents_______________________________________________
%   > ...
%
\usepackage{etoolbox}

\makeatletter
\pretocmd{\contentsline}
  {\patchcmd{\cftdotfill}
     {\leaders}
     {\hyper@linkstart{link}{#4}\leaders}
     {}
     {}%
   \patchcmd{\cftdotfill}
     {\hfill}
     {\hfill\hyper@linkend}
     {}
     {}}
  {}
  {}
\makeatother
%%%
%%% Absatz Einstellungen
%%%%%%%%%%%%%%%%%%%%%%%%%%%%%%%%%%%%%%%%%%%%%%%%%%%%%%%%%%%%
%%%%%%%%%%%                Absätze               %%%%%%%%%%%
%%%%%%%%%%%%%%%%%%%%%%%%%%%%%%%%%%%%%%%%%%%%%%%%%%%%%%%%%%%%
%
% ____ Absätze _____________________________________________
%

\RedeclareSectionCommand[
  beforeskip=-2.5\baselineskip,
  afterskip=1\baselineskip]{chapter}
  
\RedeclareSectionCommand[
  beforeskip=-0.5\baselineskip,
  afterskip=0.25\baselineskip]{section}
  
\RedeclareSectionCommand[
  beforeskip=-.75\baselineskip,
  afterskip=.5\baselineskip]{subsection}
  
\RedeclareSectionCommand[
  beforeskip=-.5\baselineskip,
  afterskip=.25\baselineskip]{subsubsection}
  
\RedeclareSectionCommand[
  beforeskip=.5\baselineskip,
  afterskip=-1em]{paragraph}
  
\RedeclareSectionCommand[
  beforeskip=-.5\baselineskip,
  afterskip=-1em]{subparagraph}
%%%
%%% Kopf- und Fußzeilen Einstellungen
%%%%%%%%%%%%%%%%%%%%%%%%%%%%%%%%%%%%%%%%%%%%%%%%%%%%%%%%%%%%%%%%%%%%%%%%%%%%
%%%%%%%%%%%                  Header and Footer                   %%%%%%%%%%%
%%%%%%%%%%%%%%%%%%%%%%%%%%%%%%%%%%%%%%%%%%%%%%%%%%%%%%%%%%%%%%%%%%%%%%%%%%%%
\usepackage[
	automark,		% Kapitelangaben in Kopfzeile automatisch erstellen
	%headtopline,   % Trennlinie über Kopfzeile
	headsepline,	% Trennlinie unter Kopfzeile
	ilines		    % Trennlinie linksbündig ausrichten
]{scrlayer-scrpage}
%\clearpairofpagestyles
%\headtopline{2pt}
%\setheadsepline{.5pt}

%%%
%%% Kopf- und Fußzeile auch auf Kapitelanfangsseiten
%\renewcommand*{\chapterpagestyle}{scrheadings}

%%%%%%%%%%%%%%%%%%%%%%%%%%%%%%%%%%%%%%%%%%%%%%%%%%%%%%%%%%%%
%%%%%%%%%%%                 Header               %%%%%%%%%%%
%%%%%%%%%%%%%%%%%%%%%%%%%%%%%%%%%%%%%%%%%%%%%%%%%%%%%%%%%%%%

% ____ Font-Style __________________________________________
\renewcommand{\headfont}{\normalfont}

%\renewcommand\chaptermark[1]{\markboth{\MakeUppercase{#1}}{}}
%\renewcommand{\chaptermark}[1]{\markboth{#1}{#1}}

%\pagestyle{scrheadings}
%\clearscrheadings
%\clearscrheadfoot
%\clearscrplain

% Automatic page numbers and column titles
%\automark[]{chapter, section}                  % column titles

%\automark[section]{chapter}
%\renewcommand*{\chaptermarkformat}{}           % No chapter number in head
%\renewcommand*{\sectionmarkformat}{}           % No section number in header

\ihead{}
\chead{}
\ohead{\textit{\leftmark \enspace}}

%\ohead{\headmark}
%\automark[subsection]{section}

%\setlength{\headheight}{21mm}                  % Header height
%\setlength{\headheight}{5mm}                   % Header height
%\headwidth[0pt]{textwithmarginpar}             % Widen header beyond text
%\headsepline[text]{0.4pt}                      % Separator line under header








%%%%%%%%%%%%%%%%%%%%%%%%%%%%%%%%%%%%%%%%%%%%%%%%%%%%%%%%%%%%
%%%%%%%%%%%                 Footer               %%%%%%%%%%%
%%%%%%%%%%%%%%%%%%%%%%%%%%%%%%%%%%%%%%%%%%%%%%%%%%%%%%%%%%%%

% ____ Footer ______________________________________________
\ifoot{}
%\cfoot{\pagemark}
\ofoot{}
%\ofoot{\includegraphics[scale=0.04]{media/Logo.png}}

%%%
%%% Bibliotheken Klasse
%%%%%%%%%%%%%%%%%%%%%%%%%%%%%%%%%%%%%%%%%%%%%%%%%%%%%%%%%%%%
%%%%%%%%%%%              Bibliography            %%%%%%%%%%%
%%%%%%%%%%%%%%%%%%%%%%%%%%%%%%%%%%%%%%%%%%%%%%%%%%%%%%%%%%%%

% ____ Bibliography-Name ___________________________________
\renewcommand{\bibname}{Literaturverzeichnis}

% ____ Bibliography-Style __________________________________
% \usepackage[square]{natbib}                     

% ____ Literature references in square brackets _____________
% \bibpunct{[}{]}{;}{a}{}{,~}

%\DefineBibliographyStrings{english}{%
%    backrefpage  = {see p.}, % for single page number
%    backrefpages = {see pp.} % for multiple page numbers
%}

\usepackage{amsmath}

%\usepackage{comment}

\usepackage[
    backend=biber
    ,style=alphabetic
    %,style=authoryear
    %
    % SORTING
    ,sorting=ynt
    %,sorting=nty           % sort by name, title, year
    %,sorting=nyt           % sort by name, year, title
    %,sorting=nyvt          % sort by name, year, volume, title
    %,sorting=anyt          % sort by alphabetic label, name, year, title
    %,sorting=anyvt         % sort by alphabetic label, name, year, volume, title
    %,sorting=ydnt          % sort by year (descending), name, title
    %,sorting=none          % entries are processed in citation order
    %
    %,sortcites=false       % some other example options ...
    %,block=none
    ,indexing=true
    %,citereset=none
    %,url=true
    %,isbn=true
    %,doi=true              % prints doi
    ,backref                % Inserts a return point in the bibliography
    %,backref=section,      % ???
    %,backref=slide,        % ???
    ,natbib=true            % if you need natbib functions
]{biblatex}

% natbib functionality
%
% command	description
% \citet{}	Textual citation
% \citep{}	Parenthetical citation
% \citet*{}	Same as \citet but if there are several authors, all names are printed
% \citep*{}	The same as \citep but if there are several authors, all names are printed
% \citeauthor{}	Prints only the name of the authors(s)
% \citeyear{}	Prints only the year of the publication.
%%%
%%% Anhang Klasse
%%%%%%%%%%%%%%%%%%%%%%%%%%%%%%%%%%%%%%%%%%%%%%%%%%%%%%%%%%%%
%%%%%%%%%%%                Appendix              %%%%%%%%%%%
%%%%%%%%%%%%%%%%%%%%%%%%%%%%%%%%%%%%%%%%%%%%%%%%%%%%%%%%%%%%

% ____ Appendix ____________________________________________
\usepackage[toc, page]{appendix}

%\usepackage{draftwatermark}
%\usepackage[firstpage]{draftwatermark}
%\SetWatermarkText{BIO-HAZARD}
%\SetWatermarkLightness{0.5}
%\SetWatermarkScale{4}

% Abkürzungsverzeichnis & Glossar ------------------------------------------
%
%
% Erstellung eines Index und Abkürzungsverzeichnisses aktivieren -----------
%\makeindex
%\makenomenclature
%
\usepackage[acronym]{glossaries}
\makeglossaries
%
%%%%%%%%%%%%%%%%%%%%%%%%%%%%%%%%%%%%%%%%%%%%%%%%%%%%%%%%%%%%%%%%%%%%%%%%%%%%
%%%%%%%%%%%%%%%%%%                  Glossar                  %%%%%%%%%%%%%%%
%%%%%%%%%%%%%%%%%%%%%%%%%%%%%%%%%%%%%%%%%%%%%%%%%%%%%%%%%%%%%%%%%%%%%%%%%%%%


\newglossaryentry{spaCy}
{
        name=spaCy,
        description={Eine Open-Source-Softwarebibliothek für die erweiterte Verarbeitung natürlicher Sprache, die in den Programmiersprachen Python und Cython geschrieben ist.}
}

\newglossaryentry{feature_extraction}
{
        name=Feature Extraction,
        description={Is a mark up language specially suited for 
scientific documents}
}

\newglossaryentry{nlp}
{
        name=NLP,
        description={Mathematics is what mathematicians do}
}

\newglossaryentry{text-mining}
{
        name=Text-Mining,
        description={A mathematical expression}
}
%%%%%%%%%%%%%%%%%%%%%%%%%%%%%%%%%%%%%%%%%%%%%%%%%%%%%%%%%%%%
%%%%%%%%%%%               Acronym                %%%%%%%%%%%
%%%%%%%%%%%%%%%%%%%%%%%%%%%%%%%%%%%%%%%%%%%%%%%%%%%%%%%%%%%%
%
% ____ Acronym _____________________________________________
%
\newacronym{safe}{SAFE}{Simple Approach for Feature Extraction}

\newacronym{acr-nlp}{NLP}{Natural Language Processing}

\newacronym{idc}{IDC}{International Data Corporation}

\newacronym{ocr}{OCR}{Optical Character Recognition}

\newacronym{po}{PO}{Product Owner}

\newacronym{pb}{PB}{Product Backlog}

\newacronym{bow}{BoW}{Bag of Words}

\newacronym{lda}{LDA}{Latent Dirichlet Allocation}

\newacronym{srs}{SRS}{Software Requirements Specification}

\newacronym{pos}{POS}{Part of Speech}

\newacronym{ide}{IDE}{Integrated Development Environment}

\newacronym{api}{API}{Application Programming Interface}

\newacronym{xp}{XP}{Extreme Programming}

\newacronym{kdd}{KDD}{Knowledge Discovery in Databases}

\newacronym{acr-pm}{PM}{Projektmanagement}

%
% --------------------------------------------------------------------------

\newcounter{blankpages}

\makeatletter

\def\cleardoubleoddstandardpage{%
\clearpage
\if@twoside \ifodd \c@page \else
 \stepcounter{blankpages}%
  \hbox {}\newpage \if@twocolumn \hbox {}\newpage \fi 
\fi \fi }
\makeatletter


\renewcommand\thepage{\the\numexpr\value{page}-\value{blankpages}\relax}

\usepackage{graphicx,eso-pic,lipsum,etoolbox}
\providecommand{\chapterhook}{}
\makeatletter
\patchcmd{\scr@startchapter}{\thispagestyle}{\chapterhook\thispagestyle}{\typeout{Patching chapter worked OK!}}{\typeout{Patching chapter failed! Oh no!!}}
\newcommand*{\chapterimage}[2][]{% \parthook[<options>]{<image>}
  \renewcommand{\chapterhook}{% Update \parthook
    \AddToShipoutPictureBG*{% Add picture to background of THIS page only
      \AtPageLowerLeft{%
        \includegraphics[width=\paperwidth,height=\paperheight,#1]{#2}%
      }%
    }% Insert image
    \renewcommand{\chapterhook}{}% Restore \parthook
  }%
}
\makeatother

\usepackage{imakeidx}
\makeindex[intoc, columns=2, title=Alphabetischer Index]

% --------------------------------------------------------------------------
%%%%%%%%%%%%%%%%%%%%%%%%%%%%%%%%%%%%%%%%%%%%%%%%%%%%%%%%%%%%%%%%%%%%%%%%%%%%
%%%%%%%%%%%%%                     Dokument                     %%%%%%%%%%%%%
%%%%%%%%%%%%%%%%%%%%%%%%%%%%%%%%%%%%%%%%%%%%%%%%%%%%%%%%%%%%%%%%%%%%%%%%%%%%
%
\begin{document}
%
%
%
% Deckblatt ----------------------------------------------------------------
%
\thispagestyle{plain}
\begin{titlepage}
\begin{center}

\begin{flushright}
    \normalsize{
        \textbf{
            \institutname\\
            \institutaddress\\
            \institutlocation
        }
    }
\end{flushright}

\singlespacing

\includegraphics[scale=0.4]{media/HS-Flensburg.png}\\[7ex]

\singlespacing

\Large{\textbf{\textsc{\art}}}\\[1.5ex]

\singlespacing

\large{\textbf{\textsc{\titlename}}}\\[3ex]

% \LARGE{\textbf{\subtitlename}}\\[4ex]
% \Large{im Studiengang \studienbereich}\\[6ex]

\singlespacing

\large{
    \authorname\\[1.5ex]
    \authoraddress\\[1.5ex]
    \authorlocation\\[1.5ex]
}

\singlespacing

\normalsize
\begin{tabular}{z{5.4cm}p{6cm}}\\
 Matrikelnummer:    & \quad \matrikelnr\\[1.2ex]
 Studiengang:       & \quad \studiengang\\[1.2ex]
 Erstgutachter:     & \quad \erstgutachter\\[1.2ex]
 Zweitgutachter:    & \quad \zweitgutachter\\[1.2ex]
 Ausgabedatum:      & \quad \ausgabedatum\\[3ex]
\end{tabular}

\singlespacing

vorgelegt am \abgabedatum \space zur Erlangung des akademischen Grades\\[1.2ex]
\akademischergrad\\[1.2ex]
im Studiengang \studiengang

\end{center}

\singlespacing

\end{titlepage}


\newpage
\null\thispagestyle{empty}\newpage % Empty Page
% --------------------------------------------------------------------------
%
% Eidesstattliche Erklärung ------------------------------------------------
%		Die eidesstattliche Erklärung befindet sich in der 001_Affidavit.tex
%
%
%%%%%%%%%%%%%%%%%%%%%%%%%%%%%%%%%%%%%%%%%%%%%%%%%%%%%%%%%%%%
%%%%%%%%%%%              Affidavit               %%%%%%%%%%%
%%%%%%%%%%%%%%%%%%%%%%%%%%%%%%%%%%%%%%%%%%%%%%%%%%%%%%%%%%%%
%
% ____ Affidavit ___________________________________________
%
\chapter*{Affidavit}
\thispagestyle{empty}

I certify that I have written this thesis independently without outside help and that I have used only the sources indicated.

\vspace{3cm}

\begin{tabular}{p{7cm}p{.5cm}l}
\hrulefill  \\
%\dotfill \\
\centering Place, Date
\end{tabular}% 
\hfill
\begin{tabular}{p{7cm}p{.5cm}l}
\hrulefill \\
%\dotfill \\
\centering Signature
%\centering\authorname
\end{tabular}% 

\newpage\null\thispagestyle{empty}\newpage % Empty Page
% --------------------------------------------------------------------------
%
\pagenumbering{roman}
\setcounter{page}{1}
%
% Vorwort ------------------------------------------------------------------
%		Hier steht das Vorwort des Dokuments
%
%%%%%%%%%%%%%%%%%%%%%%%%%%%%%%%%%%%%%%%%%%%%%%%%%%%%%%%%%%%%
%%%%%%%%%%%               Preword                %%%%%%%%%%%
%%%%%%%%%%%%%%%%%%%%%%%%%%%%%%%%%%%%%%%%%%%%%%%%%%%%%%%%%%%%
%
% ____ Preword _____________________________________________
%
\chapter*{Vorwort}
%\addcontentsline{toc}{chapter}{Vorwort}
\thispagestyle{empty}

\epigraph{Niemand meint alles was er sagt.}{Unbekannt}

\lipsum[2-3]

\newpage\null\thispagestyle{empty}\newpage % Empty Page
% --------------------------------------------------------------------------
%
% Abstract -----------------------------------------------------------------
%		Hier steht das Abstract bzw. die Zusammenfassung des Dokuments
%
%%%%%%%%%%%%%%%%%%%%%%%%%%%%%%%%%%%%%%%%%%%%%%%%%%%%%%%%%%%%%%%%%%%%%%%%%%%%
%%%%%%%%%%%%%%%%%%                Abstract                   %%%%%%%%%%%%%%%
%%%%%%%%%%%%%%%%%%%%%%%%%%%%%%%%%%%%%%%%%%%%%%%%%%%%%%%%%%%%%%%%%%%%%%%%%%%%
\chapter*{Abstract}
%\addcontentsline{toc}{chapter}{Abstract}
\thispagestyle{empty}
Text-Mining: Auswertung unstrukturierter (textueller) Informationen.

Pendent dazu wäre das Data-Mining: Auswertung strukturierter Informationen.

Aus den unterschiedlichen Untersuchungsgegenständen, ergeben sich jedoch andere Schwerpunkte bei der Vorverarbeitung:

- Beim Data-Mining steht die Bereinigung, Normalisierung der Daten und der Verbindung unterschiedlichster Daten im MiOelpunkt.

- Beim Text-Mining steht die Erkennung und ExtrakKon repräsentaKver Konstrukte im MiOelpunkt

Disziplinen

Clustering, Klassifikation, Mustererkennung, Information Retrieval, Domänenwissen

Folie 01-15

Hauptaufgaben des Text-Mining Prozesses nach Zhang und Segall (2008)

oder 

Hauptaufgaben des Text-Mining Prozesses nach Hotho et al. (2005)

\begin{itemize}
    \item Ein Problem identifizieren und analysieren.
    \item Verschiedene Lösungsansätze identifizieren, gegenüberstellen und bezüglich des Problems bewerten.
    \item Den gewählten Lösungsansatz umsetzen und dokumentieren.
\end{itemize}


\newpage\null\thispagestyle{empty}\newpage % Empty Page
% --------------------------------------------------------------------------
%
% Dankesagung --------------------------------------------------------------
%		Hier steht die Dankesagung des Dokuments
%
%%%%%%%%%%%%%%%%%%%%%%%%%%%%%%%%%%%%%%%%%%%%%%%%%%%%%%%%%%%%
%%%%%%%%%%%           Acknowledgement            %%%%%%%%%%%
%%%%%%%%%%%%%%%%%%%%%%%%%%%%%%%%%%%%%%%%%%%%%%%%%%%%%%%%%%%%
%
% ____ Acknowledgement _____________________________________
%
\chapter*{Acknowledgement}
%\addcontentsline{toc}{chapter}{Acknowledgement}
\thispagestyle{empty}
%\thispagestyle{plain}

\lipsum[1]

\vspace{0.5cm}

\begin{displayquote}
\textit{No one means everything they say, and few say everything they think. Words are slippery and thoughts are sticky. - Henry Adams}
\end{displayquote}

\newpage
\newpage\null\thispagestyle{empty}\newpage % Empty Page
% --------------------------------------------------------------------------
%
%
% Inhaltsverzeichnis -------------------------------------------------------
%
%
% Table of Contents --------------------------------------------------------
%		Das Inhaltsverzeichnis in all seiner Pracht.
%
% --------------------------------------------------------------------------

\setcounter{secnumdepth}{4}         % Tiefe der Nummerierung der Kapitel
\setcounter{tocdepth}{3}            % Tiefe des Inhaltsverzeichnis

\makeatletter
\renewcommand*\l@section{\@dottedtocline{1}{1.5em}{2.3em}}
\renewcommand*\l@subsection{\@dottedtocline{2}{3.8em}{3.2em}}
\renewcommand*\l@subsubsection{\@dottedtocline{3}{7.0em}{4.1em}}
\renewcommand*\l@paragraph{\@dottedtocline{4}{10em}{5em}}
\renewcommand*\l@subparagraph{\@dottedtocline{5}{12em}{6em}}
\makeatother

%\RedeclareSectionCommand[style=section, beforeskip=-3.5ex plus -1ex minus -.2ex, afterskip=2.3ex plus.2ex,
%   font=\normalfont\LARGE\bfseries, indent=0pt, tocindent=0em, tocnumwidth=1.5em]{part}
 
%\RedeclareSectionCommand[tocindent=1em, tocnumwidth=1em]{chapter}
%\RedeclareSectionCommand[tocindent=1em, tocnumwidth=2em]{section}
%\RedeclareSectionCommand[tocindent=2em, tocnumwidth=1.5em]{subsection}
%\RedeclareSectionCommand[tocindent=3em, tocnumwidth=1.5em]{subsubsection}
 
%\DeclareSectionCommand[level=4,style=section,
%   beforeskip=-3.25ex plus -1ex minus -.2ex, afterskip=1.5ex plus .2ex,
%   font=\normalfont\normalsize\bfseries, counterwithin=subsubsection,
%   indent=0pt,tocindent=4em, tocnumwidth=1.5em]{subsubsubsection}




%\thispagestyle{empty}                                  % disable numbering
\addtocontents{toc}{\protect\thispagestyle{empty}}      % more then 1 page
\renewcommand*{\tableofcontents}{\listoftoc[{\contentsname}]{toc}}% ToC under control of tocbasic
\AfterTOCHead[toc]{\thispagestyle{empty}\pagestyle{empty}}
\AfterStartingTOC[toc]{\clearpage}
%
\cleardoublepage
\phantomsection
%
%\addcontentsline{toc}{chapter}{Inhaltsverzeichnis}     % Inhaltsverzeichnis zum Inhaltsverzeichnis hinzufügen
\renewcommand{\contentsname}{Inhaltsverzeichnis}
\tableofcontents    % Inhaltsverzeichnis
%
%
% --------------------------------------------------------------------------
%
% Abbildungsverzeichnis -------------------------------------------------------
%
%
\renewcommand{\listfigurename}{Abbildungsverzeichnis}
\listoffigures      % Abbildungsverzeichnis
\pagenumbering{Roman}
% --------------------------------------------------------------------------
%
% Tabellenverzeichnis ------------------------------------------------------
%
%
\renewcommand{\listtablename}{Tabellenverzeichnis}
\listoftables       % Tabellenverzeichnis
% --------------------------------------------------------------------------
%
% Abkürzungsverzeichnis ----------------------------------------------------
%
%
\clearpage
\printglossary[title=Abkürzungsverzeichnis, type=\acronymtype]
% --------------------------------------------------------------------------
%
% Glossar ------------------------------------------------------------------
%
%
\printglossary[title=Glossar]
\newpage\null\thispagestyle{empty}\newpage % Empty Page
% --------------------------------------------------------------------------
%
% Listings-Verzeichnis -----------------------------------------------------
%
%
%\renewcommand{\lstlistlistingname}{Verzeichnis der Listings}
%\lstlistoflistings  % Listings-Verzeichnis
% --------------------------------------------------------------------------
%
%
%
%
% ...danach in normalen arabischen Ziffern ---------------------------------
\clearpage
\pagenumbering{arabic}
\setcounter{page}{1}
%
%
%
% Inhalt -------------------------------------------------------------------
%		Hier können jetzt die einzelnen Kapitel inkludiert werden. Sie müssen
%		in den entsprechenden .TEX-Dateien vorliegen. Die Dateinamen können
% 	    natürlich angepasst werden.
%
\chapterimage{media/headerLogo.png}
\chapter{Introduction}

\lipsum[3-5]

\section{Background}

\lipsum[3-5]

\subsection{General thoughts on the subject}

\lipsum[3-5]

\subsection{Motivation}

\lipsum[3-5]

\subsection{Ziele der Arbeit}

\lipsum[3-5]

\subsection{Vorgehen}

\lipsum[3-5]

\chapterimage{media/headerLogo.png}
\chapter{Software Requirements Specification}

\lipsum[3-5]

\section{Lastenheft}

\lipsum[3-5]

\section{Pflichtenheft}

\lipsum[3-5]

\section{SCRUM}

\lipsum[3-5]

\chapterimage{media/ChapterBackground.png}
\chapter{Vergleichende Arbeiten}

\lipsum[3-16] 

\section{SAFE-Approach}

\lipsum[2-7] 

\section{Deep Learning}

NLP\index{Sprache!nlp}

spaCy\index{Sprache!spaCy}

NN\index{Technologie!nn}

spaCy\index{Sprache|see{Sprachverarbeitung}}

NLTK\index{Sprache|seealso{NLTK}}

\chapter{Methoden}

\lipsum[3-5]

\section{Forschungsfragen}

\lipsum[3-5]

\section{Kontext}

\lipsum[3-5]

\section{Realisation}

\lipsum[3-5]

\section{Evaluation}

\lipsum[3-5]

\chapterimage{media/ChapterBackground.png}
\chapter{Feature Engineering für NLP}

\lipsum[2-3]

\section{Text Vorverarbeitung}

\lipsum[2-3]

\subsection{Satz Tokenisierung}

\lipsum[2-3]

\subsection{Stoppwörter entfernen}

\lipsum[2-3]

\subsection{Part-of-Speech tagging}

\lipsum[3-5]

\subsection{Lastenheft}

\subsubsection{Pflichtenheft}

\lipsum[3-5]

\subsubsection{DIN 69901-5}

\lipsum[3-5]

\subsection{Product Backlog}

\lipsum[3-5]

\chapterimage{media/ChapterBackground.png}
\chapter{Ergebnis}

\lipsum[3-14]

\chapterimage{media/ChapterBackground.png}
\chapter{Zusammenfassung und Fazit}

\begin{figure}
    \centering
    \caption{Caption}
    \label{fig:my_label}
\end{figure}

The table \ref{table:1} is an example of referenced \LaTeX elements.

\begin{table}[h!]
\centering
\begin{tabular}{||c c c c||} 
 \hline
 Col1 & Col2 & Col2 & Col3 \\ [0.5ex] 
 \hline\hline
 1 & 6 & 87837 & 787 \\ 
 2 & 7 & 78 & 5415 \\
 3 & 545 & 778 & 7507 \\
 4 & 545 & 18744 & 7560 \\
 5 & 88 & 788 & 6344 \\ [1ex] 
 \hline
\end{tabular}
\caption{Table to test captions and labels}
\label{table:1}
\end{table}

\input{content/08}
\input{content/09}
% --------------------------------------------------------------------------
%
%
\cleardoublepage
\newpage
\pagenumbering{Roman}
%
% Literaturverzeichnis -----------------------------------------------------
%		Das Literaturverzeichnis wird aus der Datenbank Bibliographie.bib 
% 	    erstellt. Die genaue Verwendung von bibtex wird hier jedoch nicht erklärt.
%		Link: http://de.wikipedia.org/wiki/BibTeX
%
\bibliographystyle{plainnat}
\bibliography{literature}
% --------------------------------------------------------------------------

% Anhang -------------------------------------------------------------------
%		Die Inhalte des Anhangs werden analog zu den Kapiteln inkludiert.
%		Dies geschieht in der Datei 002_Appendix.tex
%
\begin{appendix}
\setcounter{page}{5}
%%%%%%%%%%%%%%%%%%%%%%%%%%%%%%%%%%%%%%%%%%%%%%%%%%%%%%%%%%%%
%%%%%%%%%%%               Appendix               %%%%%%%%%%%
%%%%%%%%%%%%%%%%%%%%%%%%%%%%%%%%%%%%%%%%%%%%%%%%%%%%%%%%%%%%
%
% ____ Appendix ____________________________________________
%
\chapter{Anhang}
\label{anhang}

\chapterimage{media/ChapterImage.png}
\section{Software used}
\label{sec:software-used}

\vspace{3cm}

\begin{savenotes}
    \begin{table}[!ht]
    \centering
        \begin{tabular}{lll}
        \rowcolor[HTML]{FFFFFF} 
        \textbf{Software} & \textbf{Kategorie} & \textbf{Version} \\ \hline
        \rowcolor[HTML]{FFFFFF} 
        Python & Programmiersprache & 3.8.3\\
        \rowcolor[HTML]{EFEFEF} 
        pdfminer.six\footnote{Webseite: \url{https://pypi.org/project/pdfminer.six/}} & Informationsextraktion & 20200517\\
        \rowcolor[HTML]{FFFFFF}
        pyLDAviz\footnote{Github: \url{https://github.com/bmabey/pyLDAvis}} & Themenmodellierung & 2.1.2\\
        \rowcolor[HTML]{EFEFEF} 
        Gensim\footnote{Webseite: \url{https://radimrehurek.com/gensim}} & Themenmodellierung & 3.8.3\\
        \rowcolor[HTML]{FFFFFF} 
        spaCy\footnote{Github: \url{https://github.com/explosion/spaCy}} & NLP & 2.2.4\\
        \rowcolor[HTML]{EFEFEF} 
        scikit-learn\footnote{Webseite: \url{https://scikit-learn.org}} & Maschinelles Lernen & 0.23.1\\
        Jupyter Lab\footnote{Webseite: \url{https://jupyter.org}} & IDE & 2.1.4\\
        \rowcolor[HTML]{EFEFEF} 
        pandas\footnote{Webseite: \url{https://pandas.pydata.org}} & Datenverwaltung & 1.0.4\\
        \rowcolor[HTML]{FFFFFF} 
        NumPy\footnote{Webseite: \url{https://numpy.org}} & wissenschaftliches Rechnen & 1.18.4\\

        \end{tabular}%
    \caption{Übersicht der verwendeten Software}
    \label{tab:used-software}
    \end{table}
\end{savenotes}

\clearpage

\section{Konfusionsmatrizen der externen Pflichtenhefte}
\label{app:pflichtenhefte}

\vspace{2cm}

%%%%%%%%%%%
%%%% PH_A01
\begin{table}[!ht]
    \begin{subtable}{0.465\textwidth}
    \centering
        \resizebox{0.60\textwidth}{!}{%
        \begin{tabular}{cc|cc}
            \multicolumn{1}{c}{} &\multicolumn{1}{c}{} &\multicolumn{2}{c}{\scriptsize Vorhergesagt} \\ 
            \multicolumn{1}{c}{} & 
            \multicolumn{1}{c|}{} & 
            \multicolumn{1}{c}{Wahr} & 
            \multicolumn{1}{c}{Falsch} \\ \hline
            \multirow[c]{2}{*}{\rotatebox[origin=tr]{90}{\tiny Tatsächlich}}
            & Wahr      & 109    & 65     \\ [1.5ex]
            & Falsch    & 36    & 24    \\ \hline
        \end{tabular}
        }
        \caption{SAFE Confusion Matrix}
        \label{cmA01safe}
    \end{subtable}
    \hspace{0.2cm}
    \begin{subtable}{0.465\textwidth}
    \centering
        \resizebox{0.60\textwidth}{!}{%
        \begin{tabular}{cc|cc}
            \multicolumn{1}{c}{} &\multicolumn{1}{c}{} &\multicolumn{2}{c}{\scriptsize Vorhergesagt} \\ 
            \multicolumn{1}{c}{} & 
            \multicolumn{1}{c|}{} & 
            \multicolumn{1}{c}{Wahr} & 
            \multicolumn{1}{c}{Falsch} \\ \hline
            \multirow[c]{2}{*}{\rotatebox[origin=tr]{90}{\tiny Tatsächlich}}
            & Wahr      & 118   & 56     \\ [1.5ex]
            & Falsch    & 22    & 38    \\ \hline
        \end{tabular}
        }
        \caption{thesis Confusion Matrix}
        \label{cmA01thesis}
    \end{subtable}
    \vspace{0.1cm}
    \begin{subtable}{0.465\textwidth}
    \centering
        \resizebox{\textwidth}{!}{%
        \begin{tabular}{
        >{\columncolor[HTML]{FFFFFF}}r 
        >{\columncolor[HTML]{FFFFFF}}r 
        >{\columncolor[HTML]{FFFFFF}}r 
        >{\columncolor[HTML]{FFFFFF}}r 
        >{\columncolor[HTML]{FFFFFF}}r }
        \multicolumn{1}{l}{\cellcolor[HTML]{FFFFFF}} &
          \multicolumn{1}{l}{\cellcolor[HTML]{FFFFFF}precision} &
          \multicolumn{1}{l}{\cellcolor[HTML]{FFFFFF}recall} &
          \multicolumn{1}{l}{\cellcolor[HTML]{FFFFFF}f1-score} &
          \multicolumn{1}{l}{\cellcolor[HTML]{FFFFFF}support} \\ \cline{2-5} 
        \multicolumn{1}{r|}{\cellcolor[HTML]{FFFFFF}}             &      &      &      & \multicolumn{1}{r|}{\cellcolor[HTML]{FFFFFF}}    \\ \cline{1-1}
        \multicolumn{1}{|r}{\cellcolor[HTML]{FFFFFF}0}            & 0.25 & 0.85 & 0.39 & \multicolumn{1}{r|}{\cellcolor[HTML]{FFFFFF}60}  \\ \hline
        \multicolumn{1}{|r}{\cellcolor[HTML]{FFFFFF}1}            & 0.73 & 0.14 & 0.23 & \multicolumn{1}{r|}{\cellcolor[HTML]{FFFFFF}174}  \\ \hline
                                                                  &      &      &      &                                                  \\ \hline
        \multicolumn{1}{|r}{\cellcolor[HTML]{FFFFFF}accuray}      &      &      & 0.32 & \multicolumn{1}{r|}{\cellcolor[HTML]{FFFFFF}234} \\
        \multicolumn{1}{|r}{\cellcolor[HTML]{FFFFFF}macro avg}    & 0.49 & 0.49 & 0.31 & \multicolumn{1}{r|}{\cellcolor[HTML]{FFFFFF}234} \\
        \multicolumn{1}{|r}{\cellcolor[HTML]{FFFFFF}weighted avg} & 0.61 & 0.32 & 0.27 & \multicolumn{1}{r|}{\cellcolor[HTML]{FFFFFF}234} \\ \hline
        \end{tabular}%
        }
        \caption{SAFE POS-Muster}
        \label{csA01safe}
    \end{subtable}
    \hspace{0.2cm}
    \begin{subtable}{0.465\textwidth}
    \centering
        \resizebox{\textwidth}{!}{%
        \begin{tabular}{
        >{\columncolor[HTML]{FFFFFF}}r 
        >{\columncolor[HTML]{FFFFFF}}r 
        >{\columncolor[HTML]{FFFFFF}}r 
        >{\columncolor[HTML]{FFFFFF}}r 
        >{\columncolor[HTML]{FFFFFF}}r }
        \multicolumn{1}{l}{\cellcolor[HTML]{FFFFFF}} &
          \multicolumn{1}{l}{\cellcolor[HTML]{FFFFFF}precision} &
          \multicolumn{1}{l}{\cellcolor[HTML]{FFFFFF}recall} &
          \multicolumn{1}{l}{\cellcolor[HTML]{FFFFFF}f1-score} &
          \multicolumn{1}{l}{\cellcolor[HTML]{FFFFFF}support} \\ \cline{2-5} 
        \multicolumn{1}{r|}{\cellcolor[HTML]{FFFFFF}}             &      &      &      & \multicolumn{1}{r|}{\cellcolor[HTML]{FFFFFF}}    \\ \cline{1-1}
        \multicolumn{1}{|r}{\cellcolor[HTML]{FFFFFF}0}            & 0.40 & 0.63 & 0.49 & \multicolumn{1}{r|}{\cellcolor[HTML]{FFFFFF}60}  \\ \hline
        \multicolumn{1}{|r}{\cellcolor[HTML]{FFFFFF}1}            & 0.84 & 0.68 & 0.75 & \multicolumn{1}{r|}{\cellcolor[HTML]{FFFFFF}174}  \\ \hline
                                                                  &      &      &      &                                                  \\ \hline
        \multicolumn{1}{|r}{\cellcolor[HTML]{FFFFFF}accuray}      &      &      & 0.67 & \multicolumn{1}{r|}{\cellcolor[HTML]{FFFFFF}234} \\
        \multicolumn{1}{|r}{\cellcolor[HTML]{FFFFFF}macro avg}    & 0.62 & 0.66 & 0.62 & \multicolumn{1}{r|}{\cellcolor[HTML]{FFFFFF}234} \\
        \multicolumn{1}{|r}{\cellcolor[HTML]{FFFFFF}weighted avg} & 0.73 & 0.67 & 0.69 & \multicolumn{1}{r|}{\cellcolor[HTML]{FFFFFF}234} \\ \hline
        \end{tabular}%
        }
        \caption{thesis POS-Muster}
        \label{csA01thesis}
    \end{subtable}%
    \caption{Konfusionsmatrizen und Kennzahlen des PH\_A01}
    \label{tabs:resultsPHA01}
\end{table}

\vspace{2cm}

%%%%%%%%%%%
%%%% PH_A02
\begin{table}[!ht]
    \begin{subtable}{0.465\textwidth}
    \centering
        \resizebox{0.60\textwidth}{!}{%
        \begin{tabular}{cc|cc}
            \multicolumn{1}{c}{} &\multicolumn{1}{c}{} &\multicolumn{2}{c}{\scriptsize Vorhergesagt} \\ 
            \multicolumn{1}{c}{} & 
            \multicolumn{1}{c|}{} & 
            \multicolumn{1}{c}{Wahr} & 
            \multicolumn{1}{c}{Falsch} \\ \hline
            \multirow[c]{2}{*}{\rotatebox[origin=tr]{90}{\tiny Tatsächlich}}
            & Wahr      & 48    & 40    \\ [1.5ex]
            & Falsch    & 37    & 18    \\ \hline
        \end{tabular}
        }
        \caption{SAFE Confusion Matrix}
        \label{cmA02safe}
    \end{subtable}
    \hspace{0.2cm}
    \begin{subtable}{0.465\textwidth}
    \centering
        \resizebox{0.60\textwidth}{!}{%
        \begin{tabular}{cc|cc}
            \multicolumn{1}{c}{} &\multicolumn{1}{c}{} &\multicolumn{2}{c}{\scriptsize Vorhergesagt} \\ 
            \multicolumn{1}{c}{} & 
            \multicolumn{1}{c|}{} & 
            \multicolumn{1}{c}{Wahr} & 
            \multicolumn{1}{c}{Falsch} \\ \hline
            \multirow[c]{2}{*}{\rotatebox[origin=tr]{90}{\tiny Tatsächlich}}
            & Wahr      & 62    & 26    \\ [1.5ex]
            & Falsch    & 23    & 32    \\ \hline
        \end{tabular}
        }
        \caption{thesis Confusion Matrix}
        \label{cmA02thesis}
    \end{subtable}
    \vspace{0.1cm}
    \begin{subtable}{0.465\textwidth}
    \centering
        \resizebox{\textwidth}{!}{%
        \begin{tabular}{
        >{\columncolor[HTML]{FFFFFF}}r 
        >{\columncolor[HTML]{FFFFFF}}r 
        >{\columncolor[HTML]{FFFFFF}}r 
        >{\columncolor[HTML]{FFFFFF}}r 
        >{\columncolor[HTML]{FFFFFF}}r }
        \multicolumn{1}{l}{\cellcolor[HTML]{FFFFFF}} &
          \multicolumn{1}{l}{\cellcolor[HTML]{FFFFFF}precision} &
          \multicolumn{1}{l}{\cellcolor[HTML]{FFFFFF}recall} &
          \multicolumn{1}{l}{\cellcolor[HTML]{FFFFFF}f1-score} &
          \multicolumn{1}{l}{\cellcolor[HTML]{FFFFFF}support} \\ \cline{2-5} 
        \multicolumn{1}{r|}{\cellcolor[HTML]{FFFFFF}}             &      &      &      & \multicolumn{1}{r|}{\cellcolor[HTML]{FFFFFF}}    \\ \cline{1-1}
        \multicolumn{1}{|r}{\cellcolor[HTML]{FFFFFF}0}            & 0.31 & 0.33 & 0.32 & \multicolumn{1}{r|}{\cellcolor[HTML]{FFFFFF}55}  \\ \hline
        \multicolumn{1}{|r}{\cellcolor[HTML]{FFFFFF}1}            & 0.56 & 0.55 & 0.55 & \multicolumn{1}{r|}{\cellcolor[HTML]{FFFFFF}88}  \\ \hline
                                                                  &      &      &      &                                                  \\ \hline
        \multicolumn{1}{|r}{\cellcolor[HTML]{FFFFFF}accuray}      &      &      & 0.46 & \multicolumn{1}{r|}{\cellcolor[HTML]{FFFFFF}144} \\
        \multicolumn{1}{|r}{\cellcolor[HTML]{FFFFFF}macro avg}    & 0.29 & 0.29 & 0.29 & \multicolumn{1}{r|}{\cellcolor[HTML]{FFFFFF}144} \\
        \multicolumn{1}{|r}{\cellcolor[HTML]{FFFFFF}weighted avg} & 0.46 & 0.46 & 0.46 & \multicolumn{1}{r|}{\cellcolor[HTML]{FFFFFF}144} \\ \hline
        \end{tabular}%
        }
        \caption{SAFE POS-Muster}
        \label{csA02safe}
    \end{subtable}%
    \hspace{0.2cm}
    \begin{subtable}{0.465\textwidth}
    \centering
        \resizebox{\textwidth}{!}{%
        \begin{tabular}{
        >{\columncolor[HTML]{FFFFFF}}r 
        >{\columncolor[HTML]{FFFFFF}}r 
        >{\columncolor[HTML]{FFFFFF}}r 
        >{\columncolor[HTML]{FFFFFF}}r 
        >{\columncolor[HTML]{FFFFFF}}r }
        \multicolumn{1}{l}{\cellcolor[HTML]{FFFFFF}} &
          \multicolumn{1}{l}{\cellcolor[HTML]{FFFFFF}precision} &
          \multicolumn{1}{l}{\cellcolor[HTML]{FFFFFF}recall} &
          \multicolumn{1}{l}{\cellcolor[HTML]{FFFFFF}f1-score} &
          \multicolumn{1}{l}{\cellcolor[HTML]{FFFFFF}support} \\ \cline{2-5} 
        \multicolumn{1}{r|}{\cellcolor[HTML]{FFFFFF}}             &      &      &      & \multicolumn{1}{r|}{\cellcolor[HTML]{FFFFFF}}    \\ \cline{1-1}
        \multicolumn{1}{|r}{\cellcolor[HTML]{FFFFFF}0}            & 0.54 & 0.58 & 0.56 & \multicolumn{1}{r|}{\cellcolor[HTML]{FFFFFF}55}  \\ \hline
        \multicolumn{1}{|r}{\cellcolor[HTML]{FFFFFF}1}            & 0.73 & 0.70 & 0.72 & \multicolumn{1}{r|}{\cellcolor[HTML]{FFFFFF}88}  \\ \hline
                                                                  &      &      &      &                                                  \\ \hline
        \multicolumn{1}{|r}{\cellcolor[HTML]{FFFFFF}accuray}      &      &      & 0.65 & \multicolumn{1}{r|}{\cellcolor[HTML]{FFFFFF}144} \\
        \multicolumn{1}{|r}{\cellcolor[HTML]{FFFFFF}macro avg}    & 0.42 & 0.43 & 0.43 & \multicolumn{1}{r|}{\cellcolor[HTML]{FFFFFF}144} \\
        \multicolumn{1}{|r}{\cellcolor[HTML]{FFFFFF}weighted avg} & 0.65 & 0.65 & 0.65 & \multicolumn{1}{r|}{\cellcolor[HTML]{FFFFFF}144} \\ \hline
        \end{tabular}%
        }
        \caption{thesis POS-Muster}
        \label{csA02thesis}
    \end{subtable}
    \caption{Konfusionsmatrizen und Kennzahlen des PH\_A02}
    \label{tabs:resultsPHA02}
\end{table}

%%%%%%%%%%%
%%%% PH_A03
\begin{table}[!ht]
    \begin{subtable}{0.465\textwidth}
    \centering
        \resizebox{0.60\textwidth}{!}{%
        \begin{tabular}{cc|cc}
            \multicolumn{1}{c}{} &\multicolumn{1}{c}{} &\multicolumn{2}{c}{\scriptsize Vorhergesagt} \\ 
            \multicolumn{1}{c}{} & 
            \multicolumn{1}{c|}{} & 
            \multicolumn{1}{c}{Wahr} & 
            \multicolumn{1}{c}{Falsch} \\ \hline
            \multirow[c]{2}{*}{\rotatebox[origin=tr]{90}{\tiny Tatsächlich}}
            & Wahr      & 153    & 102     \\ [1.5ex]
            & Falsch    & 108     & 139    \\ \hline
        \end{tabular}
        }
        \caption{SAFE Confusion Matrix}
        \label{cmA03safe}
    \end{subtable}
    \hspace{0.2cm}
    \begin{subtable}{0.465\textwidth}
    \centering
        \resizebox{0.60\textwidth}{!}{%
        \begin{tabular}{cc|cc}
            \multicolumn{1}{c}{} &\multicolumn{1}{c}{} &\multicolumn{2}{c}{\scriptsize Vorhergesagt} \\ 
            \multicolumn{1}{c}{} & 
            \multicolumn{1}{c|}{} & 
            \multicolumn{1}{c}{Wahr} & 
            \multicolumn{1}{c}{Falsch} \\ \hline
            \multirow[c]{2}{*}{\rotatebox[origin=tr]{90}{\tiny Tatsächlich}}
            & Wahr      & 202   & 53    \\ [1.5ex]
            & Falsch    & 81    & 166    \\ \hline
        \end{tabular}
        }
        \caption{thesis Confusion Matrix}
        \label{cmA03thesis}
    \end{subtable}
    \vspace{0.1cm}
    \begin{subtable}{0.465\textwidth}
    \centering
        \resizebox{\textwidth}{!}{%
        \begin{tabular}{
        >{\columncolor[HTML]{FFFFFF}}r 
        >{\columncolor[HTML]{FFFFFF}}r 
        >{\columncolor[HTML]{FFFFFF}}r 
        >{\columncolor[HTML]{FFFFFF}}r 
        >{\columncolor[HTML]{FFFFFF}}r }
        \multicolumn{1}{l}{\cellcolor[HTML]{FFFFFF}} &
          \multicolumn{1}{l}{\cellcolor[HTML]{FFFFFF}precision} &
          \multicolumn{1}{l}{\cellcolor[HTML]{FFFFFF}recall} &
          \multicolumn{1}{l}{\cellcolor[HTML]{FFFFFF}f1-score} &
          \multicolumn{1}{l}{\cellcolor[HTML]{FFFFFF}support} \\ \cline{2-5} 
        \multicolumn{1}{r|}{\cellcolor[HTML]{FFFFFF}}             &      &      &      & \multicolumn{1}{r|}{\cellcolor[HTML]{FFFFFF}}    \\ \cline{1-1}
        \multicolumn{1}{|r}{\cellcolor[HTML]{FFFFFF}0}            & 0.58 & 0.56 & 0.57 & \multicolumn{1}{r|}{\cellcolor[HTML]{FFFFFF}247}  \\ \hline
        \multicolumn{1}{|r}{\cellcolor[HTML]{FFFFFF}1}            & 0.59 & 0.60 & 0.59 & \multicolumn{1}{r|}{\cellcolor[HTML]{FFFFFF}255}  \\ \hline
                                                                  &      &      &      &                                                  \\ \hline
        \multicolumn{1}{|r}{\cellcolor[HTML]{FFFFFF}accuray}      &      &      & 0.58 & \multicolumn{1}{r|}{\cellcolor[HTML]{FFFFFF}502} \\
        \multicolumn{1}{|r}{\cellcolor[HTML]{FFFFFF}macro avg}    & 0.58 & 0.58 & 0.58 & \multicolumn{1}{r|}{\cellcolor[HTML]{FFFFFF}502} \\
        \multicolumn{1}{|r}{\cellcolor[HTML]{FFFFFF}weighted avg} & 0.58 & 0.58 & 0.58 & \multicolumn{1}{r|}{\cellcolor[HTML]{FFFFFF}502} \\ \hline
        \end{tabular}%
        }
        \caption{SAFE POS-Muster}
        \label{csA03safe}
    \end{subtable}%
    \hspace{0.2cm}
    \begin{subtable}{0.465\textwidth}
    \centering
        \resizebox{\textwidth}{!}{%
        \begin{tabular}{
        >{\columncolor[HTML]{FFFFFF}}r 
        >{\columncolor[HTML]{FFFFFF}}r 
        >{\columncolor[HTML]{FFFFFF}}r 
        >{\columncolor[HTML]{FFFFFF}}r 
        >{\columncolor[HTML]{FFFFFF}}r }
        \multicolumn{1}{l}{\cellcolor[HTML]{FFFFFF}} &
          \multicolumn{1}{l}{\cellcolor[HTML]{FFFFFF}precision} &
          \multicolumn{1}{l}{\cellcolor[HTML]{FFFFFF}recall} &
          \multicolumn{1}{l}{\cellcolor[HTML]{FFFFFF}f1-score} &
          \multicolumn{1}{l}{\cellcolor[HTML]{FFFFFF}support} \\ \cline{2-5} 
        \multicolumn{1}{r|}{\cellcolor[HTML]{FFFFFF}}             &      &      &      & \multicolumn{1}{r|}{\cellcolor[HTML]{FFFFFF}}    \\ \cline{1-1}
        \multicolumn{1}{|r}{\cellcolor[HTML]{FFFFFF}0}            & 0.75 & 0.67 & 0.71 & \multicolumn{1}{r|}{\cellcolor[HTML]{FFFFFF}247}  \\ \hline
        \multicolumn{1}{|r}{\cellcolor[HTML]{FFFFFF}1}            & 0.71 & 0.78 & 0.74 & \multicolumn{1}{r|}{\cellcolor[HTML]{FFFFFF}255}  \\ \hline
                                                                  &      &      &      &                                                  \\ \hline
        \multicolumn{1}{|r}{\cellcolor[HTML]{FFFFFF}accuray}      &      &      & 0.73 & \multicolumn{1}{r|}{\cellcolor[HTML]{FFFFFF}502} \\
        \multicolumn{1}{|r}{\cellcolor[HTML]{FFFFFF}macro avg}    & 0.73 & 0.73 & 0.73 & \multicolumn{1}{r|}{\cellcolor[HTML]{FFFFFF}502} \\
        \multicolumn{1}{|r}{\cellcolor[HTML]{FFFFFF}weighted avg} & 0.73 & 0.73 & 0.73 & \multicolumn{1}{r|}{\cellcolor[HTML]{FFFFFF}502} \\ \hline
        \end{tabular}%
        }
        \caption{thesis POS-Muster}
        \label{csA03thesis}
    \end{subtable}%
    \caption{Konfusionsmatrizen und Kennzahlen des PH\_A03}
    \label{tabs:resultsPHA03}
\end{table}

%%%%%%%%%%%
%%%% PH_A04
\begin{table}[!ht]
    \begin{subtable}{0.465\textwidth}
    \centering
        \resizebox{0.60\textwidth}{!}{%
        \begin{tabular}{cc|cc}
            \multicolumn{1}{c}{} &\multicolumn{1}{c}{} &\multicolumn{2}{c}{\scriptsize Vorhergesagt} \\ 
            \multicolumn{1}{c}{} & 
            \multicolumn{1}{c|}{} & 
            \multicolumn{1}{c}{Wahr} & 
            \multicolumn{1}{c}{Falsch} \\ \hline
            \multirow[c]{2}{*}{\rotatebox[origin=tr]{90}{\tiny Tatsächlich}}
            & Wahr      & 19     & 13     \\ [1.5ex]
            & Falsch    & 33     & 48    \\ \hline
        \end{tabular}
        }
        \caption{SAFE Confusion Matrix}
        \label{cmA04safe}
    \end{subtable}
    \hspace{0.2cm}
    \begin{subtable}{0.465\textwidth}
    \centering
        \resizebox{0.60\textwidth}{!}{%
        \begin{tabular}{cc|cc}
            \multicolumn{1}{c}{} &\multicolumn{1}{c}{} &\multicolumn{2}{c}{\scriptsize Vorhergesagt} \\ 
            \multicolumn{1}{c}{} & 
            \multicolumn{1}{c|}{} & 
            \multicolumn{1}{c}{Wahr} & 
            \multicolumn{1}{c}{Falsch} \\ \hline
            \multirow[c]{2}{*}{\rotatebox[origin=tr]{90}{\tiny Tatsächlich}}
            & Wahr      & 18    & 14     \\ [1.5ex]
            & Falsch    & 12    & 69    \\ \hline
        \end{tabular}
        }
        \caption{thesis Confusion Matrix}
        \label{cmA04thesis}
    \end{subtable}
    \vspace{0.1cm}
    \begin{subtable}{0.465\textwidth}
    \centering
        \resizebox{\textwidth}{!}{%
        \begin{tabular}{
        >{\columncolor[HTML]{FFFFFF}}r 
        >{\columncolor[HTML]{FFFFFF}}r 
        >{\columncolor[HTML]{FFFFFF}}r 
        >{\columncolor[HTML]{FFFFFF}}r 
        >{\columncolor[HTML]{FFFFFF}}r }
        \multicolumn{1}{l}{\cellcolor[HTML]{FFFFFF}} &
          \multicolumn{1}{l}{\cellcolor[HTML]{FFFFFF}precision} &
          \multicolumn{1}{l}{\cellcolor[HTML]{FFFFFF}recall} &
          \multicolumn{1}{l}{\cellcolor[HTML]{FFFFFF}f1-score} &
          \multicolumn{1}{l}{\cellcolor[HTML]{FFFFFF}support} \\ \cline{2-5} 
        \multicolumn{1}{r|}{\cellcolor[HTML]{FFFFFF}}             &      &      &      & \multicolumn{1}{r|}{\cellcolor[HTML]{FFFFFF}}    \\ \cline{1-1}
        \multicolumn{1}{|r}{\cellcolor[HTML]{FFFFFF}0}            & 0.89 & 0.41 & 0.56 & \multicolumn{1}{r|}{\cellcolor[HTML]{FFFFFF}81}  \\ \hline
        \multicolumn{1}{|r}{\cellcolor[HTML]{FFFFFF}1}            & 0.37 & 0.88 & 0.52 & \multicolumn{1}{r|}{\cellcolor[HTML]{FFFFFF}32}  \\ \hline
                                                                  &      &      &      &                                                  \\ \hline
        \multicolumn{1}{|r}{\cellcolor[HTML]{FFFFFF}accuray}      &      &      & 0.54 & \multicolumn{1}{r|}{\cellcolor[HTML]{FFFFFF}113} \\
        \multicolumn{1}{|r}{\cellcolor[HTML]{FFFFFF}macro avg}    & 0.63 & 0.64 & 0.54 & \multicolumn{1}{r|}{\cellcolor[HTML]{FFFFFF}113} \\
        \multicolumn{1}{|r}{\cellcolor[HTML]{FFFFFF}weighted avg} & 0.74 & 0.54 & 0.55 & \multicolumn{1}{r|}{\cellcolor[HTML]{FFFFFF}113} \\ \hline
        \end{tabular}%
        }
        \caption{SAFE POS-Muster}
        \label{csA04safe}
    \end{subtable}%
    \hspace{0.2cm}
    \begin{subtable}{0.465\textwidth}
    \centering
        \resizebox{\textwidth}{!}{%
        \begin{tabular}{
        >{\columncolor[HTML]{FFFFFF}}r 
        >{\columncolor[HTML]{FFFFFF}}r 
        >{\columncolor[HTML]{FFFFFF}}r 
        >{\columncolor[HTML]{FFFFFF}}r 
        >{\columncolor[HTML]{FFFFFF}}r }
        \multicolumn{1}{l}{\cellcolor[HTML]{FFFFFF}} &
          \multicolumn{1}{l}{\cellcolor[HTML]{FFFFFF}precision} &
          \multicolumn{1}{l}{\cellcolor[HTML]{FFFFFF}recall} &
          \multicolumn{1}{l}{\cellcolor[HTML]{FFFFFF}f1-score} &
          \multicolumn{1}{l}{\cellcolor[HTML]{FFFFFF}support} \\ \cline{2-5} 
        \multicolumn{1}{r|}{\cellcolor[HTML]{FFFFFF}}             &      &      &      & \multicolumn{1}{r|}{\cellcolor[HTML]{FFFFFF}}    \\ \cline{1-1}
        \multicolumn{1}{|r}{\cellcolor[HTML]{FFFFFF}0}            & 0.83 & 0.85 & 0.84 & \multicolumn{1}{r|}{\cellcolor[HTML]{FFFFFF}81}  \\ \hline
        \multicolumn{1}{|r}{\cellcolor[HTML]{FFFFFF}1}            & 0.60 & 0.56 & 0.58 & \multicolumn{1}{r|}{\cellcolor[HTML]{FFFFFF}32}  \\ \hline
                                                                  &      &      &      &                                                  \\ \hline
        \multicolumn{1}{|r}{\cellcolor[HTML]{FFFFFF}accuray}      &      &      & 0.77 & \multicolumn{1}{r|}{\cellcolor[HTML]{FFFFFF}113} \\
        \multicolumn{1}{|r}{\cellcolor[HTML]{FFFFFF}macro avg}    & 0.72 & 0.71 & 0.71 & \multicolumn{1}{r|}{\cellcolor[HTML]{FFFFFF}113} \\
        \multicolumn{1}{|r}{\cellcolor[HTML]{FFFFFF}weighted avg} & 0.77 & 0.77 & 0.77 & \multicolumn{1}{r|}{\cellcolor[HTML]{FFFFFF}113} \\ \hline
        \end{tabular}%
        }
        \caption{thesis POS-Muster}
        \label{csA04thesis}
    \end{subtable}
    \caption{Konfusionsmatrizen und Kennzahlen des PH\_A04}
    \label{tabs:resultsPHA04}
\end{table}

\clearpage

\section{Ergebnisse der Themenmodellierung}

\vspace{4cm}

\begin{figure}[!ht]
     \centering
     \begin{subfigure}[b]{0.495\textwidth}
         \centering
         \includegraphics[width=\textwidth]{media/cs_PH_A01.png}
         \caption{PH\_01}
         \label{fig:lda-a01}
     \end{subfigure}
     \hfill
     \begin{subfigure}[b]{0.495\textwidth}
         \centering
         \includegraphics[width=\textwidth]{media/cs_PH_A02.png}
         \caption{PH\_02}
         \label{fig:lda-a02}
     \end{subfigure}
     \hfill
     \begin{subfigure}[b]{0.495\textwidth}
         \centering
         \includegraphics[width=\textwidth]{media/cs_PH_A03.png}
         \caption{PH\_03}
         \label{fig:lda-a03}
     \end{subfigure}
     \hfill
     \begin{subfigure}[b]{0.495\textwidth}
         \centering
         \includegraphics[width=\textwidth]{media/cs_PH_A04.png}
         \caption{PH\_04}
         \label{fig:lda-a04}
     \end{subfigure}
    \caption{Ergebnisse der Themenmodellierung: PH\_A01-04}
    \label{app:figs:tmtopiccoherence-external}
\end{figure}

\section{USB-Massenspeicher}

\vspace{0.5cm}

Auf dem beigelegten USB-Massenspeicher befinden sich:

\begin{itemize}
    \item Bachelorarbeit im PDF-Format
    \item Programmcode
    \begin{itemize}
        \item Jupyter Lab Projekt
        \item PyCharm Projekt
    \end{itemize}
\end{itemize}


\end{appendix}
% --------------------------------------------------------------------------

% Index --------------------------------------------------------------------
%		Zum Erstellen eines Index, die folgende Zeile auskommentieren.
%
\printindex		% Index hier einfügen
% --------------------------------------------------------------------------

\end{document}
% --------------------------------------------------------------------------
%
%          .----.__
%         / c  ^  _`;
%         |     .--'
%          \   (
%          /  -.\
%         / .   \
%        /  \    |
%       ;    `-. `.
%       |      /`'.`.
%       |      |   \ \
%       |    __|    `'
%       ;   /   \
%      ,'        |
%     (_`'---._ /--,
%       `'---._`'---..__
%              `''''--, )
%                _.-'`,`
%                 ````
%